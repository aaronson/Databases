\section{Introducci\'on}
\label{sec:introduccion}

El proceso de dise\~no de las relaciones de una base de datos involucra muchos conceptos
tanto te\'oricos como pr\'acticos relacionados con los diferentes aspectos involucrados; 
no solamente es necesario definir que entidades entrar\'an en juego y las interrelaciones
entre si, sino tambi\'en tener en cuenta aspectos de redundancia de datos y de definici\'on de
los esquemas de forma tal que se aprovechen las interrelaciones existentes.

La creaci\'on de esquemas de bases de datos a nivel industrial solamente exige ciertos
 niveles de formalidad y de correctitud, pero en el aspecto te\'orico existen muchos estudios
que definen mejoras tanto para la reducci\'on de redundancia de datos y dependencias como para
la verificaci\'on de correctitud en la definici\'on de los esquemas y las interrelaciones.

% El presente trabajo presenta un desarrollo completo del dise�o de las entidades y esquemas
 % de una base de datos basada en un caso de uso ficticio pero con posibles aplicaciones en el mundo real.
