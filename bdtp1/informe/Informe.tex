\documentclass[a4paper,10pt, spanish]{article}

%%%%%%%%%% COMIENZO DEL PREAMBULO %%%%%%%%%%

%%%%%% DATOS INICIALES
%\author{...., Federico A. Ocampo}
%\title{Trabajo Pr'actico 1 Bases de Datos}

%%%% PAQUETES

%\usepackage{infostyle}  %%Se reemplaza por los siguientes.

\usepackage[top=3cm, bottom=3cm, left=3cm, right=3cm]{geometry}  % márgenes
\usepackage[utf8]{inputenc}                 % permite que los acentos del estilo áéíóú salgan joya
\usepackage[spanish, activeacute]{babel}    % idioma español, acentos fáciles y deletreo de palabras
\selectlanguage{spanish}
\usepackage{indentfirst}                    % permite indentar un parrafo a mano
\usepackage{wrapfig}                        % permite wrappear una figura
\usepackage{graphicx}                       % permite insertar gráficos
\usepackage{color}                          % permite el uso de colores en el documento
\usepackage{listings}                       % formatea código fuente
\usepackage{ulem}                           % Linea de puntos piola
% \usepackage[section]{algorithm}
% \usepackage{algpseudocode}                  % formatea pseudocod
\usepackage{caratula}                       % incluye caratula estándar
\usepackage[font=footnotesize,labelfont=normal]{caption} %modifica apariencia de los captions

%Extras de AMS para matematicas
% \usepackage{amsthm}     %ambiente para Teoremas
% \usepackage{amsmath}    %ambientes y comandos (ej: align, eqref)
% \usepackage{amsfonts}   %fuentes
% \usepackage{amssymb}    %Simbolos matematicos

% \ifpdf
% \usepackage[pdfcreator={Kile(r) bajo GNU/Linux},
            % pdfauthor={Grupo: 16},
            % pdftitle={TP MetNum},
            % pdfsubject={Trabajo Practico de Metodos Numericos},
            % pdfkeywords={Biseccion,Newton,Aritmetica Finita},
            % pdfstartview=FitH,            % Fits the width of the page to the window
            % bookmarksnumbered,            % los bookmarks numerados se ven mejor...
            % colorlinks,                   % links con bellos colores
            % linkcolor=blue]               % permite cambiar el color de los links
            % {hyperref}                    % Permite jugar con el PDF
% \fi

%%%%% FORMATO
\linespread{1.3}                          % interlineado equivalente al 1.5 líneas de Word...
\parskip=1.5ex                            % Un poco mas de espacio entre parrafos

\pagestyle{myheadings}              %encabezado personalizable con \markboth{}{}
\markboth{}{Bases de Datos - TP1}
\headsep = 30pt                     % separación entre encabezado y comienzo del párrafo

%define el lenguaje de prog a usar
\lstset{language=SQL}
\lstset{frame=single}

%%%%%% MACROS & COMANDOS
% macro 'todo' para To-Do's
\def\TODO#1{\textcolor{red}{#1}}
\newcommand{\real}{\hbox{\bf R}}
\newcommand{\entero}{\hbox{\bf Z}}

%inicializa el ambiente Teorema
% \newtheorem{theorem}{Teorema}[section]
% \newtheorem{propo}{Proposici\'on}[section]
% \newtheorem{definition}{Definici\'on}[section]

% Macro 'borde' para un texto con borde
\newsavebox{\fmbox}
\newenvironment{borde}[1]
{\begin{lrbox}{\fmbox}\begin{minipage}{#1}}
{\end{minipage}\end{lrbox}\fbox{\usebox{\fmbox}}\\[10pt]}

%%%%%%%%%% FIN DEL PREAMBULO %%%%%%%%%%

\begin{document}

\include{Caratula}
\section{Introducci\'on}
\label{sec:introduccion}

El proceso de dise\~no de las relaciones de una base de datos involucra muchos conceptos
tanto te\'oricos como pr\'acticos relacionados con los diferentes aspectos involucrados; 
no solamente es necesario definir que entidades entrar\'an en juego y las interrelaciones
entre si, sino tambi\'en tener en cuenta aspectos de redundancia de datos y de definici\'on de
los esquemas de forma tal que se aprovechen las interrelaciones existentes.

La creaci\'on de esquemas de bases de datos a nivel industrial solamente exige ciertos
 niveles de formalidad y de correctitud, pero en el aspecto te\'orico existen muchos estudios
que definen mejoras tanto para la reducci\'on de redundancia de datos y dependencias como para
la verificaci\'on de correctitud en la definici\'on de los esquemas y las interrelaciones.

% El presente trabajo presenta un desarrollo completo del dise�o de las entidades y esquemas
 % de una base de datos basada en un caso de uso ficticio pero con posibles aplicaciones en el mundo real.

\include{Diagrama}
\include{ModeloRelacional}
\include{Funcionalidades}
\begin{thebibliography}{Databases}

\bibitem[RG97]{Ramakrishnan1997}
Raghu Ramakrishnan and Johannes Gehrke.
\newblock {\em Database Management Systems}.
\newblock McGrawHill, second edition edition, 1997.

\bibitem[DB07]{Modelizacion}
C{\'a}tedra Bases de Datos.
\newblock Apunte de modelizaci{\'o}n.
\newblock Apunte de Modelizaci{\'o}n, Marzo 2009.

\bibitem[TYF86]{Teory1986}
Toby~J. Teory, Dongqing Yang, and James~P. Fry.
\newblock A logical design metodology for relational databases using the
  extended entity-relationship model.
\newblock {\em Computing Surveys}, 18(2), jun 1986.

\bibitem[MY09]{MySQL Web Site}
\newblock MySQL Official Documentation.
\newblock http://dev.mysql.com.

\end{thebibliography}


% \appendix
% \include{ApendiceA} %Enunciado 
\end{document}
